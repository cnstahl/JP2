\documentclass[12pt]{article} % 

\usepackage[hyperfootnotes=false]{hyperref}
\usepackage[margin=1in]{geometry}                                              
\usepackage{amsmath,amsthm,amssymb}                                            
\usepackage{graphicx}                                                          
\graphicspath{{../data/}}
\usepackage{titlesec}                                                   
\usepackage{bm}
\usepackage{cprotect}
\usepackage{ bbold }
\usepackage{abstract}
\usepackage{biblatex}
\bibliography{bib.bib}
\usepackage{tikz}
\usetikzlibrary{arrows,shapes,trees} 


%%%%%%%%%%%%%%%%%%%%%%%%%%%%%%%%%%%%%%%%%%%%%%%%%%%%%%%
% Taken from Kitaev Unpaired fermions...
\newenvironment{subfig}[1]%
{\def\subfigLabel{#1}\begin{tabular}[b]{@{}c@{}}}%
	{\\[10pt]\subfigLabel\end{tabular}}

\newsavebox{\TempBox}
\newlength{\TempLength}
%%%%%%%%%%%%%%%%%%%%%%%%%%%%%%%%%%%%%%%%%%%%%%%%%%%%%%%

       
\usepackage{listings}
\lstset{basicstyle=\ttfamily,breaklines=true}

\titleformat{\subsection}[runin]
{\normalfont\large\bfseries}{\thesubsection}{1em}{}

\renewcommand{\bf}{\mathbf}
\renewcommand{\cal}{\mathcal}
\newcommand{\pd}[2]{\frac{\partial #1}{\partial #2}}
\newcommand{\pdn}[3]{\frac{\partial^{#3} #1}{\partial #2^{#3}}}
\newcommand{\pdop}[1]{\frac{\partial}{\partial #1}}
\newcommand{\nd}[2]{\frac{d #1}{d #2}}
\newcommand{\ndn}[3]{\frac{d^{#3} #1}{d #2^{#3}}}
\newcommand{\ndop}[1]{\frac{d}{d #1}}
\newcommand{\dt}{\frac{d}{dt}}
\newcommand{\grad}{\bm\nabla}
\newcommand{\cross}{\times}
\newcommand{\curl}{\grad\cross}
\newcommand{\imp}{\Longrightarrow\quad}
\newcommand{\abs}[1]{\left|#1\right|}
\newcommand{\half}{\frac{1}{2}}
\newcommand{\third}{\frac{1}{3}}
\renewcommand{\th}[1]{\frac{1}{#1}}
\renewcommand{\k}{4\pi\epsilon_0}
\newcommand{\eps}{\epsilon_0}
\newcommand{\intt}{\int_{t_1}^{t_2}}
\newcommand{\inti}{\int_{-\infty}^{+\infty}}
\newcommand{\ex}[1]{\left\langle #1 \right\rangle}
\renewcommand{\d}{\delta}
\newcommand{\e}{\text{e}}
\renewcommand{\l}{\ell}
\newcommand{\om}{\omega}
\newcommand{\h}{\hbar}
\newcommand{\ket}[1]{\left|#1\right\rangle}
\newcommand{\bra}[1]{\left\langle#1\right|}
\newcommand{\braket}[2]{\left\langle#1\middle|#2\right\rangle}
\newcommand{\brakett}[3]{\left\langle#1\middle|#2\middle|#3\right\rangle}
\newcommand{\comm}[2]{\left[#1,#2\right]}
\newcommand{\acom}[2]{\left\{#1,#2\right\}}
\newcommand{\nn}{\nonumber\\}

\DeclareMathOperator{\Tr}{Tr}

%\renewcommand{\thesection}{\arabic{section}}

\begin{document}

\title{\textbf{Majorana Fermions}}
\author{Charles Stahl}

\maketitle

\begin{abstract}
	This paper explores the ground states of a supersymmetric generalization of the Sachdev-Ye-Kitaev model, with $N$ interacting Majorana Fermions. The model consists of $N$ Majorana fermions interacting through a Hamiltonian that connects fermions in groups of 4. The supersymmetric model defines a supercharge that connects 3 fermions, with the Hamiltonian being the square of the supercharge. The number of ground states is large in this model, leading to extensive entropy. Numerical simulations of the number of ground states and entropy confirm the calculations for small $N$. 
\end{abstract}

\tableofcontents\footnote{Include or not?}
\newpage


%
%\section{Introduction} 
%
%To discuss Majorana fermions, it is helpful to discuss Dirac fermions, the first realization of relativistic fermions.
%
%\subsection{Dirac Equation}\footnote{How relevant is this? Should I cut some of it out?} \emph{}
%
%The Dirac equation can be derived by trying to mirror the nonrelativistic Shr\"odinger equation while obeying relativity. This derivation follows reference~\cite{gottfried03}. A problem with the nonrelativistic Shr\"odinger equation is that it does not treat time and space equally, in that it is first order in time and second order in space
%\begin{align}
%i\pdop{r}\psi(\bm r, t) &= \left[\frac{-1}{2m}\grad^2 +V(\bm r,t)\right]
%	\psi(\bm r, t). \label{eqn:shro}
%\end{align}
%Using the replacements
%\begin{align}
%\bm p \leftrightarrow \th{i}\grad, \quad E \leftrightarrow \th{i}\pdop{t}, 
%	\label{eqn:repl}
%\end{align}
%equation~\ref{eqn:shro} becomes $E = p^2/2m +V$, which is true in classical mechanics. In relativistic mechanics, the energy-momentum relationship becomes $E^2 = m^2 +p^2$, which reduces to the classical equation in the limit $p<<m$. Using the replacements from equation~\ref{eqn:repl}, this is
%\begin{align}
%\frac{\partial^2}{\partial t^2}\psi(\bm r,t)=\left(\grad^2-m^2\right)\psi
%	(\bm r,t),\label{eqn:klein}
%\end{align}
%the Klein Gordon equation. There are two problems with this equation as it is. It is second order in time, and does not admit of a probability interpretation.
%
%The form suggests we might want to find an operator that we could call 
%\begin{align}
%\sqrt{\grad^2-m^2}
%\end{align}
%using the Taylor series of $\sqrt{\cdot}$ which, when applied twice to $\psi$, would produce
%\begin{align}
%(\grad^2-m^2)\psi.
%\end{align}
%This attempt does not end up working because the square root includes terms of arbitrarily high order~\cite{sakurai11}.
%
%However, using the partial derivative on Minkowski space 
%\begin{align}
%\partial_\mu = (\partial_t,\grad),\quad\partial^\mu = \eta^{\mu\nu}\partial_
%	\nu = (\partial_t, -\grad),\label{eqn:partial}
%\end{align}
%we can rewrite equation~\ref{eqn:klein} as 
%\begin{align}
%\left(\partial_\mu\partial^\mu + m^2\right)\psi &= \left(\eta^{\mu\nu}\partial_
%	\mu\partial_\nu + m^2\right)\psi.
%\end{align}
%Evidently, we want a solution of the form
%\begin{align}
%(i\gamma^\mu\partial_\mu - m)\psi = 0,
%\end{align}
%so that, by operating with the complex conjugate,
%\begin{align}
%(-i\gamma^\mu\partial_\mu - m)(i\gamma^\nu\partial_\nu - m)\psi =(\gamma^\mu
%	\gamma^\nu\partial_\mu\partial_\nu + m^2) = 0. \footnote{Is this satisfying?}
%\end{align}
%Note that partial derivatives with different indices commute
%
%Since the partial derivative commutes, $\partial_\mu\partial_\nu$ is symmetric, so we can require that the $\gamma$ term above is symmetric as well.
%\begin{align}
%\gamma^\mu\gamma^\nu = \half\left(\gamma^\mu\gamma^\nu + \gamma^\nu\gamma^\mu
%	\right) = \{\gamma^\mu,\gamma^\nu\} = \eta^{\mu\nu}
%\end{align}
%This antisymmetry requirement is central to the Dirac equation.\footnote{This derivation is a mix of Gottfried and Sakurai. I think Gottfried has a better derivation in a different place, but I haven't been able to understand it. I plan to read it again this weekend and have a better writeup for this section.}
%
%\subsection{Antiparticles}
%
%\subsection{Fermion Second Quantization}\emph{}
%
%Before fermionic second quantization it is convenient to consider the bosonic case. Bosonic fields such as the electromagnetic field can be decomposed into modes $\bm p$ with raising and lowering operators $a^\dagger_{\bm{p}}$ and $a_{\bm{p}}$. Since the number of particles is not fixed, the Hilbert space spans subspaces of different $N'$. This is called the Fock Space $\mathfrak{F}$,
%\begin{align}
%\mathfrak{F} = \mathfrak{H}_0 \oplus \mathfrak{H}_1 \oplus \mathfrak{H}_2 
%	\oplus \cdots
%\end{align}
%where $\mathfrak{H}_N'$ is the subspace with $N'$ particles. 
%
%The vacuum state is written as $\ket{0}$. A state with one particle in mode $\bm p_1$ is $a^\dagger_{\bm p_1}\ket{0} = \ket{\bm p_1}$. Since bosonic states must be symmetric, the state $\ket{\bm p_1\bm p_2} = a^\dag_{\bm p_1} a^\dag_{\bm p_2}\ket{0}$ must equal to $\ket{\bm p_2\bm p_1} = a^\dag_{\bm p_2} a^\dag_{\bm p_1}\ket{0}$. From this we derive the boson commutation relations
%\begin{align}
%[a^\dag_{\bm p}, a^\dag_{\bm p'}] = [a_{\bm p}, a_{\bm p'}]=0.
%\end{align}
%The only nonzero commutation is 
%\begin{align}
%[a_{\bm p},a^\dag_{\bm p'}] = \d_{\bm{p,p}'}
%\end{align}
%
%Since these operators create and annihilate particles, they can define a number operator $N_{\bm p} = a^\dagger_{\bm p}a_{\bm p}$ with eigenvalues $N'_{\bm p}$ which count the particles in mode $\bm p$. The total number of particles is $N' = \sum_{\bm p}N'_{\bm p}$.\footnote{Is it correct to introduce the number operator as a result of the commutation relations (are the commutation relations the fundamental part)?}
%
%Second quantization for fermions must be fundamentally different. Only one fermion can exist in any mode. A fermionic state is then a series of bits; 1 if a particle exists in the corresponding mode and 0 if the mode is unoccupied. Furthermore, the state must be antisymmetric. Therefore the state
%\begin{align}
%\ket{\alpha_{\nu_1}\alpha_{\nu_2}\alpha_{\nu_3}\cdots\alpha_{\nu_N}}
%\end{align}
%is shorthand for 
%\begin{align}
%\ket{\alpha_{\nu_1}\alpha_{\nu_2}\alpha_{\nu_3}\cdots\alpha_{\nu_N}}_A = 
%	\epsilon^{ijk\cdots n}\ket{\alpha_{\nu_i}\alpha_{\nu_j}\alpha_{\nu_k}\cdots
%	\alpha_{\nu_n}}.
%\end{align}
%Here, the $\nu_i$ are some quantum numbers (not necessarily momentum modes). $\epsilon$ is the totally antisymmetric tensor with $N$ indices, where $n$ is the $N$th index.
%
%%  Anticommutation relations??
%
%This means that, when defining operators in terms of $\psi$ operators, it is convenient to define them with the operators in canonical order, for example
%\begin{align}
%\cal A = \sum_{i<j<k}A^{ijk}a_ia_ja_k.
%\end{align}
%If $A$ is defined to be antisymmetric, 
%\begin{align}
%\cal A = \th{N!}\sum_{ijk}A^{ijk}a_ia_ja_k,
%\end{align}
%where $N$ is the number of operators. 

\section{Introduction}

The Sachdev-Ye-Kitaev (SYK) model is a model of random uncorrelated interactions between $N$ fermions. Sachdev and Ye introduced a model in 1993 with pairwise interactions in~\cite{sachdev93}. The holographic connection of this model was discussed by Sachdev in 2010~\cite{sachdev10}. Kitaev generalized the model to interactions between 4 fermions, with Hamiltonian
\begin{align}
H = \sum_{ijkl}\psi_i\psi_j\psi_k\psi_l
\end{align}
in~\cite{kitaev15}. For more discussion of the model see~\cite{mald16}.

The model is built of Majorana fermions, which are fermions which are their own antiparticles~\cite{elliott14}. Solutions to the Dirac equation
\begin{align}
(\gamma^\mu\partial_\mu - m)\Phi = 0
\end{align}
are 4-component spinors. Two degrees of freedom are due to the choice of helicity, while the other two differentiate fermions from antifermions. The Majorana basis allows the equation to be separated into two independent coupled systems. Each of these is then a Majorana fermion. 

Although no Majorana particles are known to exist, the Majorana algebra can describe emergent quasiparticles in solid-state physics. Since the electron is a Dirac fermion, it is equivalently two interacting Majorana fermions. If these can be separated, the resultant quasiparticles are Majorana fermions. One way of doing this is on a one dimensional quantum ``wire." Majorana fermion pairs are separated and paired with fermions from adjacent electrons. See figure~\ref{fig:pairs} for a graphical description. This process is described in detail in~\cite{kitaev00}.
\begin{figure}[ht]
	
\sbox{\TempBox}{
\begin{picture}(0,0)
\put(0,0){\oval(32,16)}
\put(-10,0){\circle*{4}}
\put(10,0){\circle*{4}}
\end{picture}}
	
\newcommand\fsite[2]{%
\begin{picture}(0,0)
\put(0,0){\usebox{\TempBox}}
\put(-16,12){\footnotesize #1}
\put(10,12){\footnotesize #2}
\end{picture}}
	
\hbox to .8\textwidth {
\hspace{.1\textwidth}
		
\begin{subfig}{a)}
\begin{picture}(170,30)(0,-8)
\put(18,0){\fsite{$c_1$}{$c_2$}}
\put(8,0){\thicklines\line(1,0){20}}
\put(75,0){\fsite{$c_3$}{$c_4$}}
\put(65,0){\thicklines\line(1,0){20}}
\put(104,0){\ldots}
\put(146,0){\fsite{$c_{2L-1}$}{$c_{2L}$}}
\put(136,0){\thicklines\line(1,0){20}}
\end{picture}
\end{subfig}
		
\hspace{.1\textwidth}
		
\begin{subfig}{b)}
\begin{picture}(170,50)(0,-8)
\put(18,0){\fsite{$c_1$}{$c_2$}}
\put(28,0){\thicklines\line(1,0){37}}
\put(75,0){\fsite{$c_3$}{$c_4$}}
\put(85,0){\thicklines\line(1,0){18}}
\put(104,0){\ldots}
\put(146,0){\fsite{$c_{2L-1}$}{$c_{2L}$}}
\put(136,0){\thicklines\line(-1,0){18}}
\end{picture}
\end{subfig}		
}
	
\caption{\textbf{Separating Dirac fermions into Majorana fermions.}  The ovals show the physical particles, while the lines show pairings. In a) there are no unpaired Majorana fermions, but in b) both $c_1$ and $c_{2L}$ are unpaired. Image from~\cite{kitaev00}.\protect\footnotemark}
\label{fig:pairs}
\end{figure}
\footnotetext{Is this ok to use?}





\section{Supersymmetry in Oscillators}

One starting point for discussing supersymmetry is through the formalism of harmonic oscillators for bosons and fermions. This also allows for the introduction of the commutation and anticommutation relations for bosons and fermions.\footnote{I spent a lot of time debating how to start talking about creation and annihilation. How should I?}

\subsection{Bosonic and Fermionic Oscillators} \emph{}

Consider a single harmonic oscillator 
\begin{align}
H = p^2 +\om^2 x^2,\quad \comm{x}{p} = \frac{i}{2} \label{eqn:harmosc}
\end{align}
with raising and lowering operators
\begin{align}
a = \sqrt{\om}x+\frac{i}{\sqrt{\om}}p,\quad a^\dag = \sqrt{\om}x+\frac{i}{
	\sqrt{\om}}p,\quad \comm{a}{a^\dag } = 1.
\end{align}
The Hamiltonian can then be written as 
\begin{align}
H = \om a^\dag a + \th{2} \equiv \om N + \th{2}.
\end{align}
After introducing a ground state $\ket{0}$ such that $a\ket{0} = 0$, the number operator $N$ counts the number of times the raising operator has been applied to a state. This shows that the raising and lowering operators add and remove energy from the system, respectively. 

The Hamiltonian for electromagnetism can be written as
\begin{align}
H = \sum_p \left(P_p^2 + \om_p^2x_p^2\right)
\end{align}
with the proper choice of $x$ and $P$. Given a collection of modes $p$ with frequencies $\om_p$ and creation and annihilation operators $a_p^\dag$ and $a_p$, the commutation relations become 
\begin{align}
[a_p, a_{p'}] = 0,\quad [a^\dag_p, a^\dag_{p'}]=0, \quad[a_p,a^\dag_{p'}] = 
	\delta_{p,p'},
\end{align}
and the Hamiltonian becomes
\begin{align}
H = \sum_p\left(\om_pN_p+\th{2}\right),
\end{align}
meaning that acting with $a_p^\dag$ adds $\om_p$ to the energy of the state.

The interpretation is that each raising operator $a_p$ adds a photon with energy $\om_p$, for which it is called the creation operator. Likewise, $a_p^\dag$ is called the annihilation operator. Then, $N_p$ counts the number of photons with energy $\om$. Since particle number is not conserved, the state space is now a  Fock Space $\mathfrak{F}$,
\begin{align}
\mathfrak{F} = \mathfrak{H}_0 \oplus \mathfrak{H}_1 \oplus \mathfrak{H}_2 
\oplus \cdots
\end{align}
where $\mathfrak{H}_N$ is the subspace with $N$ particles. Actually, it a product of Fock spaces, one for each value of $p$.\footnote{Is this true?}

States with definite values of $N_p$ are stationary states, and can be specified by the number of particles in each mode $\{n_p\}$. Since the space is a product over $p$ of spaces for each mode, these states can be written as
\begin{align}
\ket{\{n_p\}} = \prod_pa_p^\dag\ket{0}.
\end{align}
Since operators for different modes commute, we have
\begin{align}
\ket{p_1p_2} = a^\dag_{p_1}a^\dag_{p_2}\ket{0} = a^\dag_{p_2} a^\dag_{p_1} \ket{0} = \ket{p_1p_2},
\end{align}
which is to say that the states are symmetric.

Fermionic operators obey the anticommutation relations
\begin{align}
\{b_p, b_{p'}\} = 0,\quad \{b^\dag_p, b^\dag_{p'}\} = 0,\quad \{b_p, b^\dag_{p'}\} = \delta_{p,p'}.
\end{align}
The analysis of the Fock space matches that for bosonic operators, except the states are are antisymmetric because
\begin{align}
\ket{p_1p_2} = b^\dag_{p_1}...
\end{align}

Unlike in the case of bosons, there cannot be an arbitrary number of fermions in a mode, but rather at most one. This can be seen by trying to raise a non-empty state: $b_p^\dag b^\dag_p\ket{0} = 0$. The operators maintain their interpretation as creation and annihilation operators, though.\footnote{What is the connection between anti-commutation and antiparticles?}

\subsection{Majorana Fermions} \emph{}

The algebra of Majorana fermions is closely related to that of normal (Dirac) fermions, with the difference being a lack of annihilation operator
\begin{align}
\{\psi_p,\psi_{p'}\} = \delta_{p,p'}.
\end{align}
This leads to antisymmetric states, like the Dirac Fermion, but allows for no antiparticles.\footnote{I still don't understand this. How do I not understand this?}

\subsection{Supersymmetry} \emph{}



\section{SYK Model}

All sums of operators are defined such that the operators are in canonical order.

\subsection{Definition} \emph{}

The SYK model is defined as having a Hamiltonian
\begin{align}
H = \sum_{ijk\l}J_{ijk\l}\psi^i\psi^j\psi^k\psi^\l,
\end{align}
where $J_{ijk\l}$ are drawn from a normal distribution. The operators obey the anticommutation relations
\begin{align}
\{\psi^i,\psi^j\} = \delta^{ij}.
\end{align}

\subsection{Supersymmetry}\emph{}

The discussion of the supersymmetric models comes from reference~\cite{fu16}. In the supersymmetric generalization, the Hamiltonian is written in terms of the supercharge
\begin{align}
Q = i\sum_{i<j<k}C_{ijk}\psi^i\psi^j\psi^k,
\end{align}
where $C_{ijk}$ are now drawn from a Gaussian with mean 0 and variance $2J/N^2$. Because the $\psi$ operators are antisymmetric, the other components of $C$ may be chosen so that $C$ is also antisymmetric. In this case 
\begin{align}
Q = \frac{i}{6}\sum_{ijk}C_{ijk}\psi^i\psi^j\psi^k,
\end{align}
with the indices no longer necessarily ordered.

The Hamiltonian is defined as
\begin{align}
H &= Q^2 = - \sum_{i<j<k}C_{ijk}\psi^i\psi^j\psi^k\sum_{\l<m<n}C_{\l mn}\psi^\l
	\psi^m\psi^n.
\end{align}
For those terms where $(i,j,k) = (\l,m,n)$, the sum becomes
\begin{align}
\sum_{i<j<k}C_{ijk}^2\psi^i\psi^j\psi^k\psi^i\psi^j\psi^k = \th{8} \sum_{i<j<k}
	C_{ijk}^2 
\end{align}
Eventually\footnote{I'm not able to derive this. What am I missing?} the Hamiltonian becomes
\begin{align}
H = E_0 + \sum_{i<j<k<\l}J_{ijk\l}\psi^i\psi^j\psi^kk\psi^\l, \label{eqn:N1def}
\end{align}
where
\begin{align}
E_0 = \sum_{i<j<k} C_{ijk}^2\;,\qquad J_{ijk\l} = -\th{8}\sum_{a} C_{a[ij}
	C_{kl]a}.
\end{align}
This shows that $E_0$ is positive, with expectation (over the values of $C_{ijk}$)
\begin{align}
\ex{E_0} &= \sum_{i<j<k}\ex{C^2_{ijk}}\nn
&= \th{6}N(N-1)(N-2)\frac{2J}{N^2}.
\end{align}
In the large $N$ limit this becomes $E_0 = NJ/3$.\footnote{This contradicts the Fu paper's assertion that $E_0\to 0$ for large $N$. What am I missing?}

\subsection{$\cal{N}=2$ Supersymmetry, Ground States}\emph{}

In $\cal N=2$ supersymmetry, two supercharges are necessary. To create these, define a new set of fermions that consists of $\psi^i$ and their conjugates $\bar{\psi_i}$. These obey the relations 
\begin{align}
\{\psi^i,\psi^j\} = 0, \quad \{\bar{\psi_i},\bar{\psi_j}\} = 0, \quad
	\{\psi^i,\bar{\psi_j}\} = \delta_j^i. \label{eqn:N2_ant}
\end{align}
The supercharges are 
\begin{align}
Q &= i\sum_{i<j<k}C_{ijk}\psi^i\psi^j\psi^k,\quad
\bar{Q} = i\sum_{i<j<k}\bar C^{ijk}\bar{\psi_i}\bar{\psi_j}\bar{\psi_k},
	\label{eqn:N2charge}
\end{align}
where the $C_{ijk}$ are complex numbers drawn from a 0-centered Gaussian such that
\begin{align}
\ex{C_{ijk}\bar C^{ijk}} = \frac{2J}{N^2}.
\end{align} 
The Hamiltonian is defined as 
\begin{align}
H &= \{Q, \bar Q\} = \cdots\nn
&= E_0 + \sum_{ijk\l}J_{ij}^{k\l}\psi^i\psi^j\bar{\psi_k}\bar{\psi_\l}.
\end{align}
Like in the $\cal N=1$ case, the components of $J$ are no longer independent. 

The operators $Q^2$ and $\bar Q^2$ are both 0. That implies that both $Q$ and $\bar Q$ are conserved, as for example
\begin{align}
[H,Q] = Q\bar QQ + \bar QQQ - QQ\bar Q + Q\bar QQ = 0.
\end{align}
This also means they preserve energy. This last fact suggests a convenient way of counting ground states. \_\_\_\_\_\_\_\_\_\_\_\_\_\_\_\_\_\_\_\_\_\_\_\_\_

\section{$\cal{N}$ = 2 Ground States and Entropy\protect\footnote{Mention cohomology?}}

The $\cal N = 2$ supersymmetric model has ``UNBROKEN SYMMETRY (or \_\_\_\_\_\_\_\_)", which leads to nonzero entropy in subsystems of ground states, even at 0\footnote{numerals vs words?} temperature. The high number of ground states and long-range correlation are intimately tied to this entropy.

\subsection{Counting Ground States} \emph{}

A convenient way to bound the number of ground states from below is to partition the Hilbert space into energy eigenspaces with states connected by the supercharges, and show that any space with an odd number of states must be made of ground states. Consider a stationary state $\ket{\phi_E}$ with energy $E$. Consider also the states $\ket{q_E} = Q\ket{\phi_E}$ and $\ket{\bar q_E}= \bar Q \ket{\phi_E}$. These states will have the same energy eigenvalue $E$, if they exist. 

The two simple cases are if $\ket{q}$ and or $\ket{\bar q}$ does not exist. If $Q$ and $\bar Q$ annihilate $\ket{\phi_E}$, then $E=0$. If $\bar Q$ annihilates $\ket{\phi_E}$ but $Q$ does not, then 
\begin{align}
H\ket{\phi_E} = (Q\bar{Q} + \bar QQ)\ket{\phi_E} = \bar QQ\ket{\phi_E} = E\ket{\phi_E}
\end{align}
and the state $\ket{f_E} = \bar QQ\ket{\phi_E}$ is just $\ket{\phi_E}$, $E>0$. The same analysis applies if $Q$ annihilates but $\bar Q$ does not.

If both $\ket{q_E}$ and $\ket{\bar q_E}$ exist and $E>0$, then
\begin{align}
H\ket{\phi_E} = E\ket{\phi_E} = (Q\bar Q + \bar QQ)\ket{\phi_E} = \ket{f_E} + 
	\ket{f'_E},
\end{align}
meaning 
\begin{align}
\ket{f_E} = \alpha\ket{\phi_E} + \beta\ket{\xi_E},\qquad \ket{f'_E} = 
	(1-\alpha)\ket{\phi_E} - \beta\ket{\xi_E}.
\end{align}
$\ket{\xi_E}$ must exist for $\ket{q_E}$ and $\ket{\bar q_e}$ to be distinct.

The previous analysis divides the space into tuplets of energy eigenstates with at most 4 members. See figure~\ref{fig:tuplets}. All tuplets contain an even number of states, except for some of the ground state tuplets. All even tuplets contain an equal number of fermionic and bosonic states.
\begin{figure}
	\centering 
	\begin{tikzpicture}[scale=.9, transform shape]
	\tikzstyle{every node} = [circle]
	\node (phi)  at (1,0) {$\ket{\phi_E}$};
	\node (q)    at (0, 2) {$Q\ket{\phi_E}$};
	\node (qbar) at (2, 2) {$\bar Q\ket{\phi_E}$};
	\draw [->]   (phi) -- (q);
	\draw [->]   (phi) -- (qbar);
	\fill (3.5,.5) circle (1pt);
	\fill (3.5,1.5) circle (1pt);
	\node (phi)  at (6,0) {$\ket{\phi_a}$};
	\node (q)    at (5, 2) {0};
	\node (qbar) at (7, 2) {0};
	\draw [->]   (phi) -- (q);
	\draw [->]   (phi) -- (qbar);
	\node (,)    at (8.5,.5){,};
	\node (phi)  at (11,0) {$\ket{\phi_b}$};
	\node (q)    at (10, 2) {$\ket{q_b}$};
	\node (qbar) at (12, 2) {0};
	\node (E)    at (11, 4) {$E\ket{\phi_b}$};
	\draw [->]   (phi) -- (q);
	\draw [->]   (phi) -- (qbar);
	\draw [->]   (qbar)-- (E);
	\end{tikzpicture}
	\caption{\textbf{Illustration of the possible configuration of energy eigenstate tuplets.} Representing the effects of $Q$ and $\bar Q$ as arrows up and to the left and right, respectively, we can see that any tuplet with an odd number of states must be a ground state tuplet.\protect\footnotemark}
	\label{fig:tuplets}
\end{figure}
\footnotetext{Is this graphic helpful?}

This provides the opportunity to count ground states using the Fermi number
\begin{align}
F = \sum_i\bar{\psi_i}\psi^i,
\end{align}
which is even for bosonic states and odd for fermionic states. For all such subspaces with non-zero energy, the Witten index 
\begin{align}
W = \Tr\left((-1)^F\right) \label{eqn:N2witten}
\end{align}
evaluates to 0. Then the trace over the whole space has contributions only from the ground states, so the number of ground states will be equal to or greater than $|W|$. Similarly, the \_\_something\_\_\footnote{What is this called? Wikipedia calls it the Witten index. Is this even important?}
\begin{align}
W = \Tr\left((-1)^F\e^{\beta H}\right)
\end{align}
is independent of temperature because $H$ is zero in the ground states. 

The Witten index is also independent of coupling, so it can be computed with $J=0$. Define the unitary operator 
\begin{align}
R^p = \e^{\frac{2\pi irF}{\hat q}},\quad R^3 = \mathbb{1}.
\end{align}
$R$ commutes with $Q$ so that $\Tr\left((-1)^FR^p\right)$ also only has contributions from the ground state. \_\_SOMEHOW\_\_,
\begin{align}
W_r = \Tr\left((-1)^FR^\frac{2\pi ir}{\hat q}\right) \equiv Z^{(N)},
\end{align}
where $Z^{(M)}$ is the $M$-particle partition function. Since $W$ is independent of the coupling, it can be turned off, in which case the partition function is the product of one particle partition functions $\left(Z^{(1)}\right)^N$ so that 
\begin{align}
W = \left(1-\e^{\frac{2\pi ip}{3}}\right)^N = e^{i\pi N\left(\frac{r}{\hat q} - \th{2}\right)}\left(2\sin\frac{\pi r}{\hat q}\right)^N
\end{align}
and the number of ground states is 
\begin{align}
D_\text{GS} \ge \abs{W} = \left(2\sin\frac{\pi r}{\hat q}\right)^N.
\end{align}
To provide the tightest bound on $D_\text{GS}$, choose $r = (\hat q+1)/2$, so that $D_\text{GS} \ge \sqrt{3}^N$, where the substitution $\hat q = 3$ has been made.\footnote{Why?}

Compare this to the total number of states $2^N$. It is clear that the number of ground states is extensive, as the number of degrees of freedom, the log of $D_\text{GS}$, scales linearly with $N$.

\subsection{Ground State Entanglement Entropy} \emph{}

The large number of ground states has an interesting effect on the entanglement entropy of a subsystem of a ground state. Classically, entropy measures uncertainty about an event or state. The nearest quantum mechanical analogue is Von Neumann entropy
\begin{align}
S = -\Tr(\rho\log\rho).\label{eqn:vnent}
\end{align}
Written in the eigenbasis of $\rho$, this becomes 
\begin{align}
S = -\sum_ip_i\log p_i,\quad \rho = \sum_i p_i\ket{i}\bra{i},
\end{align}
which mimics the classical form.

The entropy of a pure state is always 0. However, subsystems of a pure state are mixed, and therefore have non-zero entropy. This entropy does not represent uncertainty, since the state of the whole system is known~\cite{janzing09}. Rather, it represents the \_\_classical information between subsystems\_\_.

Consider state $\ket{\phi}_\cal{A}$ on a system $\cal A$ that can be decomposed into subsystems $A$ an $B$.\footnote{system or space?} The state can in general be written as 
\begin{align}
\ket{\phi}_\cal{A} = \sum_jc_j\ket{\phi_j}_A\ket{\phi_j}_B,
\end{align}
where the $\ket{\phi_j}$ are orthogonal in each subsystem. Since the reduced density matrices\footnote{Should I introduce reduced density matrices? This i a general question.} are
\begin{align}
\rho_A = \sum_jp_j\ket{\phi_j}_A\bra{\phi_j}_A, \quad \rho_B = \sum_jp_j\ket{\phi_j}_B\bra{\phi_j}_B; \quad p_j = |c_j|^2,
\end{align}
It is clear that the entropies of entanglement are equal for each subsystem. In fact, they satisfy 
\begin{align}
S(\rho) = S(\rho_B) = \cal{H}(p_j),
\end{align}
where $\cal H$ is the classical Shannon entropy with $p_j$ interpreted as classical probabilities. 

A general ground state in the $\cal N=2$ SYK model can be written as 
\begin{align}
\ket{\phi_0} = \sum_ic_i\ket{i_A}\ket{i_B}\footnote{notation?}
\end{align}

\section{Numerical Results}

\subsection{Building Matrices}\emph{}

The simplest way to compute large-N functions on a computer is to use a matrix representation for the group. This can be done by viewing the $\bar \psi$ and $\psi$ operators as raising and lowering operators, respectively. Consider the case of $N$ fermions. Write the state as an $N$-dimensional binary vector \footnote{I want to include this because I spent a lot of time on it. Is that ok?}
\begin{align}
\ket{m} = \ket{m_1m_2\dots m_i\dots m_{N-1}m_N}, \quad\sum_nm_n2^{n-1} =
	m.\label{eqn:2Nstate}
\end{align}
The vacuum state is $\ket{0}$. A populated state can be written as
\begin{align}
\ket{110\dots 0} = \bar\psi_1\bar{\psi_2}\ket{0} =-\bar\psi_2\bar\psi_1\ket{0},
\end{align}
which enforces the canonical ordering through anticommutation relations. 

The matrix representations of $\psi^i, \bar \psi_i$ are then calculated component-wise using an $2^N$ by $2^N$ matrix.
\begin{align}
\bar\psi_i &= \sum_{a,b}\ket{a}\bar\psi_{i,ab}\bra{b}\nn
\bar\psi_{i,ab} &= \brakett{a}{\bar\psi_i}{b}.\label{eqn:comps}
\end{align}
These components may be calculated effectively using the pseudocode in figure~\ref{code:psibar}.\footnote{Should I explain the code more?}

\begin{figure}[ht]
	\begin{lstlisting}[language=python][gobble=2]
    def psi_bar(N,i):
      psi_bar = np.zeros((2^N,2^N))
      for (n,m) in psi_bar:
        if (m-n == 2^j) and (n & 2^j == 0): 
          psi[m,n] = (-1)^Fermi(n,j),
    return psi_bar
		\end{lstlisting}
	\cprotect\caption{\textbf{Code for generating $\bar \psi$ matrices,} where \verb|^| represents exponentiation, \verb|&| represents bitwise and \verb|Fermi(n,j)|, the reduced Fermi number, is the number of nonzero components in \verb|n| before the \verb|j|'th component. This handles anticommutation requirements. A similar structure can be used for building the $\psi$ matrices.}
	\label{code:psibar}
\end{figure}

It is also possible to create the matrices recursively. The base case is $N=1$, for which the matrices are the raising and lowering matrices,
\begin{align}
\bar\psi_{0,(1)} = \begin{bmatrix} 0&0\\1&0 \end{bmatrix}, \qquad
    \psi^0_{(1)} = \begin{bmatrix} 0&1\\0&0 \end{bmatrix}. \label{eqn:base}
\end{align}
Then, to increase the dimension take the tensor product with the $2\times 2$ identity matrix $I$,
\begin{align}
\bar\psi_{i,(N)} = I\otimes\bar\psi_{i,(N-1)},\quad \psi^i_{(N)} = I\otimes 
	\psi^i_{(N-1)}.
\end{align}
This formula of course does not work when $i=N$. In this case the solution is still to use recursion, with the formula
\begin{align}
\bar\psi_{i,(N)} = \bar\psi_{i,(N-1)}\otimes\sigma_3,
\end{align}
with equation~\ref{eqn:base} again supplying the base case.

Although there are various representations of the group, the previous two constructions lead to the same representation. 

\subsection{Ground States} \emph{}

Counting ground states was implemented by counting the zero eigenvalues of a randomly-generated Hamiltonian. The number of ground states stayed above the bound, although the bound was not tight in some places. See figure~\ref{fig:gserr} for the ground states above the bound.\footnote{Would this graph be better as actual/predicted?}

\begin{figure}
	\centering
	\includegraphics[width=.5\textwidth]{gserr}
	\caption{\textbf{Looseness of bound on number of ground states.} 0 represents the bound being tight while 1 would mean twice as many ground states as the bound. Since the number of ground states is an integer, these values were calculated as $\text{ceil}(\sqrt{3}^N)$.}
	\label{fig:gserr}
\end{figure}

It is particularly interesting to note that the number of ground states is particularly high for even $N$. Even with randomly generated Hamiltonians, the relative number of ground states above the bound stays relatively constant near 0.33. 

\subsection{Ground State Entanglement Entropy}

Once the number of ground states were calculated for a single Hamiltonian, the entropy of a subsystem\footnote{Terminology?} of the ground state could be calculated from the reduced density matrix of the subsystem. Starting with a ground state $\ket{\phi}$, the density matrix $\rho = \ket{\phi}\bra{\phi}$ can be calculated. The entropy $S = \Tr\rho\log\rho$ can be calculated as 
\begin{align}
S = \sum \lambda\log\lambda,
\end{align}
where $\lambda$ are the eigenvalues. 

As predicted, the entropy rises and drops linearly as more particles are traces over. Since the entropy does not depend on the coupling, a single computation of the entropy should be sufficient.\footnote{Right?} However, due to small numerical errors, it is helpful to perform multiple calculations and average the entropies. 

To see the symmetric linear dependence of the entropy, see figure~\ref{fig:N11pred_ent}.

\begin{figure}
	\centering
	\includegraphics[width=.5\textwidth]{N11pred_ent}
	\caption{\textbf{Entropy for $N=11$.} The $x$ axis is $M$, the number of particles traced out. The solid line shows the value of entropy calculated from $S=\frac{2}{3}m$, where $m = \min(M,N-M)$ is the number of particles in the smaller subspace. }
	\label{fig:N11pred_ent}
\end{figure}

\printbibliography

\end{document}
